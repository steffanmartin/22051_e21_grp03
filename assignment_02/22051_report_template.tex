
\documentclass[journal]{IEEEtran}
%\usepackage{lineno}
%\linenumbers
%\usepackage{cite}
\usepackage{comment}
\ifCLASSINFOpdf
  \usepackage[pdftex]{graphicx}
  \graphicspath{pics/}
  \DeclareGraphicsExtensions{.pdf,.jpeg,.png,.jpg}
\else

\fi
\usepackage{amsmath}
\usepackage{siunitx}
\usepackage{geometry}
\usepackage{comment}
\usepackage{float}
\usepackage[caption=false]{subfig}
\usepackage[super]{nth}
\hyphenation{}

\begin{document}
\newgeometry{top=12mm, bottom=12mm, left=12mm, right=12mm}
\title{22051 - Assignment number 1}
\author{\vspace{-2mm}{\normalsize \underline{Group 3} - s194048 Erik Rame (\_ hrs), s194051 Jakob Bernhardt (\_ hrs), s194006 Steffan Kunoy (\_ hrs), \\s186083 Tjark Petersen (\_ hrs)}\vspace{-10mm}}

% make the title area
\maketitle
%\IEEEpeerreviewmaketitle

\section{How finite should it be?}

\subsection{Like a hot knife through a Butterworth}
% Erik
% Jakob
The purpose of this exercise was to construct a bandpass filter which met a set of requirements. These requirements were:
\begin{itemize}
\item Passband from 0.2 to 0.3
\item Stopbands from 0 to 0.1 and 0.4 to 1
\item Ripples are OK if they don't exceed 2 dB
\item The stopband attenuation should be at least -100 dB
\end{itemize}

It was decided to make a Butterworth bandpass filter. MATLAB has a built-in function called 'butter' which returns the transfer function of a Butterworth filter of the specified type, order and cutoff frequency. 
\newline
The requirements meant that we constructed a bandpass Butterworth filter using a higher and lower cutoff frequency of 0.2 and 0.3 respectively. The order was determined by testing with different values, and choosing the lowest order where the stopband attenuation was at least -100 dB. 


\subsection{Shorter and shorter}
% Steffan
For this exercise we were tasked with designing an \textit{ideal} FIR lowpass filter based on the following specifications: 
\begin{itemize}
    \item Pass-band from 0 to 5 kHz with gain of 0 dB. 
    \item Cut-off frequency at one-third the normalized frequency. 
    \item Impulse response is 601 samples long. 
\end{itemize}
From these specifications we can argue that the sampling frequency should be 30 kHz, as the Nyquist frequency would then be 15 kHz, which would correspond to three times the cut-off frequency of 5 kHz. 
\newline
We can then construct the impulse response $h$ of the filter by a sequence of $601/3$ 1's followed by $\lfloor \frac{601}{} \rfloor$
\section{An equalizer - without buttons}

\subsection{Bass and treble}
% Tjark

\end{document}


